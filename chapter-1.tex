\documentclass{article}
\usepackage{amssymb}
\usepackage{amsmath}
\usepackage{gensymb}
\usepackage{enumerate}
\usepackage[
nonumberlist, %do not show page numbers
nopostdot,    %do not add additional periods
acronym,      %generate acronym listing
toc,          %show listings as entries in table of contents
section]      %use section level for toc entries
{glossaries}
\usepackage[automake]{glossaries-extra}

\title{Sentential Logic}
\author{Axel Sorenson}
\date{April 20th, 2024}

\loadglsentries{chapter-1-glossary}
\makeglossaries

\newcommand{\printGls}[1]{%
    \textbf{\Gls{#1}} - \glsentrydesc{#1}%
}
\newcommand{\printGlspl}[1]{%
    \textbf{\Glspl{#1}} - \glsentrydesc{#1}%
}

\begin{document}
\maketitle

\section{Introduction}
Before we get into setential logic, let's establish the basics.\\
\begin{itemize}
	\item \printGlspl{conjecture}
	\item \printGlspl{theorem}
	\item \printGlspl{deductive reasoning}
\end{itemize}

\section{Deductive Reasoning and Logical Connectives}
The correctness of deductive reasoning relies on the \textit{conclusion} of a statement being forced by the \textit{premises}. Therefore, an argument is valid if the premises cannot all be true without the conclusion being true as well.\\\\
\noindent
The following is what a set of premises and a conclusion typically looks like.\\

\begin{itemize}
    \item[] I will go to work either tomorrow or today.
    \item[] I'm going to stay home today.
    \item[$\therefore$] I will go to work tomorrow.
\end{itemize}

\noindent The \gls{symb:therefore} symbol means "therefore."\\\\
\noindent
This is a valid argument because in order for the two given premises to be true, the conclusion must also be true. Since I'm going to stay home today and I will go to work either tomorrow or today, that must mean I have to go to work tomorrow.\\\\
\noindent
Some definitions:
\begin{itemize}
	\item \printGls{argument}
	\item \printGls{premise}
	\item \printGls{conclusion}
	\item \printGlspl{connective symbol}
\end{itemize}

Here are three connective symbols:
\begin{center}
\begin{tabular}{@{}cc@{}}
	\underline{Symbol} & \underline{Meaning} \\
	\gls{symb:or} & or \\
	\gls{symb:and} & and \\
	\gls{symb:not} & not 
\end{tabular}
\end{center}

Some definitions:\\
\begin{itemize}
	\item \printGls{conjunction}
	\item \printGls{disjunction}
	\item \printGls{negation}
	\item \printGls{well-formed formula}
\end{itemize}
\noindent
Something to note: mathematical expressions such as $3 \leq \pi$ are actually disjunctions: $(3 < \pi) \lor (3 = \pi)$.\\
$3 \leq \pi < 4$ is a conjunction: $[(3 < \pi) \lor (3 = \pi)] \land (\pi < 4)$.

\subsection{Exercises}
Q1. Analyze the logical forms of the following statements:
\begin{enumerate}[(a)]
	\item We’ll have either a reading assignment or homework problems, but we won’t have both homework problems and a test.
	\item You won’t go skiing, or you will and there won’t be any snow.
	\item $\sqrt{7} \nleq 2$.
\end{enumerate}
A1.
\begin{enumerate}[(a)]
\item
P = we have a reading assignment.\\
Q = we have homework problems.\\
R = we have a test.\\
$(P \lor Q) \land \lnot(Q \land R)$.
\item
P = you will go skiing\\
Q = there will be snow\\
$\lnot P \lor (P \land \lnot Q)$.
\item
P = $\sqrt{7} < 2$\\
Q = $\sqrt{7} = 2$\\
$\lnot(P \lor Q)$.\\
$\lnot[(\sqrt{7} < 2) \lor (\sqrt{7} = 2)]$.
\end{enumerate}

\clearpage
\printglossary[type=\acronymtype,style=long]  % list of acronyms
\printglossary[type=symbolslist,style=long]   % list of symbols
\printglossary[type=main]                     % main glossary
\end{document}
