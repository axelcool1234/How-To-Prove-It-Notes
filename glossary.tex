\newglossary[slg]{symbolslist}{syi}{syg}{Symbols}
% ==== MAIN GLOSSARY ====
\newglossaryentry{conjecture}{
	name={conjecture},
	description={what mathematicians call their guesses, similarly to how scientists call their guesses hypotheses.}
}

\newglossaryentry{theorem}{
	name={theorem},
	description={conjectures that have been proven. A theorem is a statement that has been proven to be true based on a set of assumptions (hypotheses). It asserts that if the hypotheses are true, then a specific conclusion must also be true.}
}

\newglossaryentry{hypothesis}{
    name={hypothesis},
    description={In the context of a theorem, a hypothesis is an assumption or a set of assumptions that are stated within the theorem. If the hypotheses are true, they support the truth of the theorem's conclusion.},
    plural={hypotheses}
}

\newglossaryentry{instance}{
    name={instance},
    description={An instance of a theorem is a specific case obtained by assigning particular values to the free variables within the theorem. For all instances where the hypotheses are true, the conclusion must also be true for the theorem to be correct.}
}

\newglossaryentry{counterexample}{
    name={counterexample},
    description={A counterexample to a theorem is a specific instance where the hypotheses of the theorem are true, but the conclusion is false. The existence of a counterexample disproves the theorem.}
}

\newglossaryentry{deductive reasoning}{
	name={deductive reasoning},
	description={the foundation on which proofs are based. Proofs play a central role in mathematics.}
}

\newglossaryentry{argument}{
	name={argument},
	description={a set of premises and a conclusion. An argument is valid if it is the case that, if the premises are true then the conclusion must be true.}
}

\newglossaryentry{premise}{
	name={premise},
	description={a fact or assumption in an argument.}
}

\newglossaryentry{conclusion}{
	name={conclusion},
	description={the end product of an argument. In a valid argument the conclusion will be true if the given premises are true.}
}

\newglossaryentry{connective symbol}{
	name={connective symbol},
	description={stand-ins for some of the words used to combine statements in an argument.}	
}

\newglossaryentry{conjunction}{
	name={conjunction},
	description={P \gls{symb:and} Q is the conjunction of P and Q.}
}

\newglossaryentry{disjunction}{
	name={disjunction},
	description={P \gls{symb:inclusive or} Q is the disjunction of P and Q.}
}

\newglossaryentry{negation}{
	name={negation},
	description={\gls{symb:not} P is the negation of P.}
}

\newglossaryentry{conditional}{
	name={conditional},
    description={P \gls{symb:implies} Q is a conditional.}
}

\newglossaryentry{well-formed formula}{
	name={well-formed formula},
	description={a "grammatical" expression in the language of logic.}
}

\newglossaryentry{free variable}{
    name={free variable},
    description={The free variables in a statement stand for objects that the statement says something about. Plugging in different values for a free variable affects the meaning of a statement and may change its truth value. The fact that you can plug in different values for a free variable means that it is free to stand for anything.}
}

\newglossaryentry{bound variable}{
    name={bound variable},
    description={The bound variables (also known as dummy variables) in a statement are simply letters that are used as a convenience to help express an idea and should not be thought of as standing for any particular object. A bound variable can always be replaced by a new variable without changing the meaning of the statement, and often the statement can be rephrased so that the bound variables are eliminated altogether.}
}

\newglossaryentry{truth set}{
    name={truth set},
    description={The set of all values that make a given satement true.}
}

\newglossaryentry{universe of discourse}{
    name={universe of discourse},
    description={The set of all possible values that \glspl{free variable} can take in a given statement. The universe of discourse is often denoted by $U$.}
}

\newglossaryentry{rational number}{
    name={rational number},
    description={A number that can be expressed as the quotient of two integers, where the denominator is not zero. Rational numbers are denoted by $\mathbb{Q}$.}
}

\newglossaryentry{irrational number}{
    name={irrational number},
    description={A number that cannot be expressed as the quotient of two integers. Irrational numbers have non-repeating, non-terminating decimal expansions and include numbers such as $\pi$ and $\sqrt{2}$.}
}

\newglossaryentry{integer number}{
    name={integer number},
    description={A number that can be written without a fractional or decimal component. Integers include the set of positive whole numbers, negative whole numbers, and zero, and are denoted by $\mathbb{Z}$.}
}

\newglossaryentry{real number}{
    name={real number},
    description={A value that represents a quantity along a continuous line. Real numbers include all rational numbers (such as fractions and integers) and all irrational numbers (numbers that cannot be expressed as a fraction of two integers), and are denoted by $\mathbb{R}$.}
}

\newglossaryentry{natural number}{
    name={natural number},
    description={A positive integer (1, 2, 3, ...) used for counting and ordering. Some definitions include zero, while others do not. Natural numbers are denoted by $\mathbb{N}$.}
}

\newglossaryentry{complex number}{
    name={complex number},
    description={A number in the form $a + bi$, where $a$ and $b$ are real numbers, and $i$ is the imaginary unit with the property that $i^2 = -1$. The real part of the complex number is $a$, and the imaginary part is $b$.}
}

\newglossaryentry{disjoint}{
    name={disjoint},
    description={Sets $A$ and $B$ are said to be disjoint if $A \cap B = \varnothing$, which means that $A$ and $B$ have no common elements.},
}

\newglossaryentry{antecedent}{
    name={antecedent},
    description={In a conditional statement $P \rightarrow Q$, the antecedent is $P$. It is the statement that comes before the arrow and is assumed to be true for the implication to hold.}
}

\newglossaryentry{consequent}{
    name={consequent},
    description={In a conditional statement $P \rightarrow Q$, the consequent is $Q$. It is the statement that comes after the arrow and is asserted to be true if the \gls{antecedent} $P$ is true.}
}

\newglossaryentry{arbitrary}{
    name={arbitrary},
    description={In the context of a proof, a variable is considered arbitrary if no specific assumptions are made about its value. An arbitrary variable can represent any element within the \gls{universe of discourse}, and the proof must hold for all possible values of this variable.}
}

% Logical Equivalences
\newglossaryentry{De Morgan's laws}{
    name={De Morgan's laws},
    description={
        $\neg(P \land Q)$ is equivalent to $\neg P \lor \neg Q$.\\
        $\neg(P \lor Q)$ is equivalent to $\neg P \land \neg Q$.
    }
}
\newglossaryentry{Commutative laws}{
    name={Commutative laws},
    description={
        $P \land Q$ is equivalent to $Q \land P$.\\
        $P \lor Q$ is equivalent to $Q \lor P$.
    }
}
\newglossaryentry{Associative laws}{
    name={Associative laws},
    description={
        $P \land (Q \land R)$ is equivalent to $(P \land Q) \land R$.\\
        $P \lor (Q \lor R)$ is equivalent to $(P \lor Q) \lor R$.
    }
}
\newglossaryentry{Idempotent laws}{
    name={Idempotent laws},
    description={
        $P \land P$ is equivalent to $P$.\\
        $P \lor P$ is equivalent to $P$.
    }
}
\newglossaryentry{Distributive laws}{
    name={Distributive laws},
    description={
        $P \land (Q \lor R)$ is equivalent to $(P \land Q) \lor (P \land R)$.\\
        $P \lor (Q \land R)$ is equivalent to $(P \lor Q) \land (P \lor R)$.
    }
}
\newglossaryentry{Absorption laws}{
    name={Absorption laws},
    description={
        $P \lor (P \land Q)$ is equivalent to $P$.\\
        $P \land (P \lor Q)$ is equivalent to $P$.
    }
}
\newglossaryentry{Double Negation laws}{
    name={Double Negation laws},
    description={
        $\neg \neg P$ is equivalent to $P$.
    }
}
\newglossaryentry{Tautology laws}{
    name={Tautology laws},
    description={Formulas that are always true are called tautologies.\\
        $P \land \text{(a tautology)}$ is equivalent to $P$.\\
        $P \lor \text{(a tautology)}$ is a tautology.\\
        $\neg \text{(a tautology)}$ is a contradiction. 
    }
}
\newglossaryentry{Contradiction laws}{
    name={Contradiction laws},
    description={Formulas that are always false are called contradictions.\\
        $P \land \text{(a contradiction)}$ is a contradiction.\\
        $P \lor \text{(a contradiction)}$ is equivalent to $P$.\\
        $\neg \text{(a contradiction)}$ is a tautology.
    }
}

\newglossaryentry{Conditional laws}{
    name={Conditional laws},
    description={
        $P \text{ \gls{symb:implies} } Q$ is equivalent to $\neg P \lor Q$.\\
        $P \text{ \gls{symb:implies} } Q$ is equivalent to $\neg (P \land \neg Q)$.
    }
}

\newglossaryentry{Contrapositive law}{
    name={Contrapositive law},
    description={
        $P \text{ \gls{symb:implies} } Q$ is equivalent to $\neg Q \rightarrow \neg P$. 
    }
}

\newglossaryentry{Quantifier Negation laws}{
    name={Quantifier Negation laws},
    description={
        $\neg \exists x P(x)$ is equivalent to $\forall x \neg P(x)$.\\
        $\neg \forall x P(x)$ is equivalent to $\exists x \neg P(x)$.\\
    }
}

\newglossaryentry{vacuously true}{
    name={vacuously true},
    description={A statement is said to be vacuously true if it asserts that all elements of the empty set have a certain property. For example, the statement "all unicorns have wings" is vacuously true because there are no unicorns. In general, a statement of the form $\forall x \in A, P(x)$ is vacuously true if $A$ is empty. Another way of looking at it is converting the universal statement that is vacuously true into a conditional statement of the form $\forall x (x \in A \rightarrow P(x))$ and realizing that the antecedent cannot be satisfied (which in our example would be the statement $x \in A$. This is because there is no $x$ at all in our set of unicorns - it is the empty set).}
}

\newglossaryentry{family of sets}{
    name={family of sets},
    description={A family of sets is a set whose elements are themselves sets. It is often indexed by another set, allowing each set in the family to be identified by a unique index. For example, $\{A_i \mid i \in I\}$ denotes a family of sets $\{A_i\}$ indexed by $I$.}
}

\newglossaryentry{power set}{
    name={power set},
    description={The power set of a set $A$, denoted by $\text{\gls{symb:power set}}(A)$ or $2^A$, is the set of all subsets of $A$, including the empty set and $A$ itself. Formally, $\mathcal{P}(A) = \{B \mid B \subseteq A\}$.}
}

\newglossaryentry{given}{
    name={given},
    description={In the context of a proof, givens are the statements that are known or assumed to be true at a particular point in the proof process. Initially, the givens include the hypotheses of the theorem being proven, but they may later include other statements inferred from the hypotheses or added as new assumptions through transformations.}
}

\newglossaryentry{goal}{
    name={goal},
    description={In the context of a proof, the goal is the statement that remains to be proven at a particular point in the proof process. Initially, the goal is the conclusion of the theorem being proven, but it may change several times as transformations are applied during the course of the proof.}
}

\newglossaryentry{existential instantiation}{
    name={existential instantiation},
    description={A logical rule that allows one to introduce a new variable to stand for an object that satisfies an existential statement. If you have a given of the form $\exists x P(x)$, you can introduce a new variable $x_0$ to represent an object for which $P(x_0)$ is true, thereby allowing the assumption that $P(x_0)$ is true.}
}

\newglossaryentry{universal instantiation}{
    name={universal instantiation},
    description={A logical rule that allows one to derive a statement about a specific element from a universally quantified statement. If you have a given of the form $\forall x P(x)$, you can substitute any specific value, say $a$, for $x$ and conclude that $P(a)$ is true, allowing you to treat $P(a)$ as a new given.}
}

\newglossaryentry{modus ponens}{
    name={modus ponens},
    description={A logical rule that states if $P \rightarrow Q$ is true and $P$ is true, then $Q$ must also be true. Also known as the rule of detachment.}
}

\newglossaryentry{modus tollens}{
    name={modus tollens},
    description={A logical rule that states if $P \rightarrow Q$ is true and $Q$ is false, then $P$ must also be false. Also known as the rule of contrapositive.}
}

\newglossaryentry{case}{
    name={case},
    description={When one of your givens is a disjunction, such as $P \lor Q$, you need to consider two possibilities: if $P$ is true and if $Q$ is true. This involves creating separate arguments for each possibility to show that both lead to the goal.},
}

\newglossaryentry{exhaustive}{
    name={exhaustive},
    description={When all possible cases are considered to ensure that every scenario is accounted for. Mathematicians consider an argument \textit{exhaustive} if it includes all potential situations, ensuring completeness of the proof.}
}

\newglossaryentry{disjunctive syllogism}{
    name={disjunctive syllogism},
    description={A rule of inference stating that if you have a given of $P \lor Q$ and also a given of $\lnot P$, then you can conclude that $Q$ is true (adding it to your givens). Similarly, if you have a given of $P \lor Q$ and also a given of $\lnot Q$, then you can conclude that $P$ is true (adding it to your givens).}
}

\newglossaryentry{coordinate}{
    name={coordinate},
    description={An element in an ordered pair or tuple. In an ordered pair $(a, b)$, $a$ is the first coordinate and $b$ is the second coordinate. The order of the coordinates makes a difference in the pair or tuple.}
}

\newglossaryentry{singleton}{
    name={singleton},
    description={A set that contains exactly one element. For example, $\{a\}$ is a singleton set containing the element $a$.}
}

\newglossaryentry{relationship}{
    name={relationship},
    description={A connection or association between two or more variables, often expressed as a statement involving those variables. For example, $P(x, y)$ can express a relationship between $x$ and $y$.}
}

\newglossaryentry{relation}{
    name={relation},
    description={Suppose $A$ and $B$ are sets. Then a set $R \subseteq A \times B$ is called a relation from $A$ to $B$. It represents a set of ordered pairs $(a, b)$ where $a \in A$ and $b \in B$, capturing when a certain relationship holds between elements of $A$ and $B$.}
}

\newglossaryentry{domain}{
    name={domain},
    description={Suppose $R$ is a relation from $A$ to $B$. The domain of $R$ is the set $\text{Dom}(R) = \{a \in A \mid \exists b \in B((a,b) \in R)\}$. It represents all elements $a$ in $A$ that are related to some element $b$ in $B$.}
}

\newglossaryentry{range}{
    name={range},
    description={Suppose $R$ is a relation from $A$ to $B$. The range of $R$ is the set $\text{Ran}(R) = \{b \in B \mid \exists a \in A((a,b) \in R)\}$. It represents all elements $b$ in $B$ that are related to some element $a$ in $A$.}
}

\newglossaryentry{inverse}{
    name={inverse},
    description={Suppose $R$ is a relation from $A$ to $B$. The inverse of $R$ is the set $R^{-1} = \{(b,a) \in B \times A \mid (a,b) \in R\}$. It represents the set of ordered pairs where the roles of the elements in $A$ and $B$ are reversed.}
}

\newglossaryentry{composition}{
    name={composition},
    description={Suppose $R$ is a relation from $A$ to $B$ and $S$ is a relation from $B$ to $C$. The composition of $S$ and $R$ is the relation $S \circ R$ from $A$ to $C$ defined as $S \circ R = \{(a,c) \in A \times C \mid \exists b \in B((a,b) \in R \text{ and } (b,c) \in S)\}$. It represents the set of ordered pairs $(a, c)$ where there exists an element $b$ such that $a$ is related to $b$ by $R$ and $b$ is related to $c$ by $S$.}
}

\newglossaryentry{relation on}{
    name={relation on},
    description={A relation $R$ from set $A$ to the same set $A$ is called a relation on $A$.},
}

\newglossaryentry{directed graph}{
    name={directed graph},
    description={A directed graph is a set of vertices connected by edges, where each edge has a direction associated with it, going from one vertex to another. This direction indicates a relationship between the vertices, often represented as an ordered pair of vertices $(a, b)$. In a directed graph, the edge $(a, b)$ is different from the edge $(b, a)$.},
}

\newglossaryentry{reflexive}{
    name={reflexive},
    description={A relation $R$ on a set $A$ is said to be reflexive if $\forall x \in A((x, x) \in R)$. In other words, every element is related to itself.},
}

\newglossaryentry{symmetric}{
    name={symmetric},
    description={A relation $R$ on a set $A$ is said to be symmetric if $\forall x \in A \forall y \in A((x, y) \in R \rightarrow (y, x) \in R)$. In other words, if one element is related to another, then the second element is related to the first.},
}

\newglossaryentry{transitive}{
    name={transitive},
    description={A relation $R$ on a set $A$ is said to be transitive if $\forall x \in A \forall y \in A \forall z \in A(((x, y) \in R \land (y, z) \in R) \rightarrow (x, z) \in R)$. In other words, if an element is related to a second element, which is related to a third element, then the first element is related to the third element.},
}

\newglossaryentry{loop}{
    name={loop},
    description={In a directed graph, a loop is an edge that connects a vertex to itself. If a relation is reflexive, its directed graph will have loops on all its vertices.},
}

\newglossaryentry{identity relation}{
    name={identity relation},
    description={For a set $A$, the identity relation $i_{A}$ is defined as $i_{A} = \{(x,y) \in A \times A \mid x = y \}$ or equivalently $i_{A} = \{(x,x) \mid x \in A \}$. The identity relation on any set $A$ is reflexive, symmetric, and transitive.},
}

\newglossaryentry{antisymmetric}{
    name={antisymmetric},
    description={A relation $R$ on a set $A$ is said to be antisymmetric if $\forall x \in A \forall y \in A ((xRy \land yRx) \rightarrow x = y)$. In other words, if $x$ is related to $y$ and $y$ is related to $x$ under the relation $R$, then $x$ must be equal to $y$.},
}

\newglossaryentry{partial order}{
    name={partial order},
    description={A relation $R$ on a set $A$ is called a partial order on $A$ (or just a partial order if $A$ is clear from context) if it is reflexive, transitive, and antisymmetric. In other words, $R$ is a partial order if for all $x, y, z \in A$:
        \begin{itemize}
            \item Reflexive: $xRx$,
            \item Transitive: $xRy \land yRz \rightarrow xRz$,
            \item Antisymmetric: $xRy \land yRx \rightarrow x = y$.
        \end{itemize}},
}

\newglossaryentry{total order}{
    name={total order},
    description={A relation $R$ on a set $A$ is called a total order on $A$ (or just a total order if $A$ is clear from context) if it is a partial order, and in addition it has the property that for all $x, y \in A$, $xRy$ or $yRx$ (i.e., the relation $R$ is comparable for every pair of elements in $A$). In other words, $R$ is a total order if it is reflexive, transitive, antisymmetric, and for any $x, y \in A$ either $xRy$ or $yRx$ holds. Note that a total order has only one minimal element, which is also the smallest element. Note that a total order has only one maximal element, which is also the largest element.},
}

\newglossaryentry{smallest}{
    name={smallest},
    description={Suppose $R$ is a partial order on a set $A$, $B \subseteq A$, and $b \in B$. Then $b$ is called an $R$-smallest of $B$ (or just a smallest element if $R$ is clear from the context) if $\forall x \in B(bRx)$. This means that $b$ is related to every other element in $B$ by the relation $R$, with $b$ on the left side, indicating it is "smaller" than every other element in $B$. Note that if there is a smallest element, it is unique. Note that it'd also be the only minimal element. Note that if there is a smallest element, it is the greatest lower bound.},
}

\newglossaryentry{minimal}{
    name={minimal},
    description={Suppose $R$ is a partial order on a set $A$, $B \subseteq A$, and $b \in B$. Then $b$ is called an $R$-minimal element (or just a minimal element if $R$ is clear from the context) if $\lnot \exists x \in B(xRb \land x \neq b)$. This means that there does not exist an element $x \in B$ such that $x$ is related to $b$ by the relation $R$ with $x$ on the left side (indicating $b$ is "bigger"), and $x$ is not equal to $b$.},
}

\newglossaryentry{largest}{
    name={largest},
    description={Suppose $R$ is a partial order on a set $A$, $B \subseteq A$, and $b \in B$. Then $b$ is called an $R$-largest of $B$ (or just a largest element if $R$ is clear from the context) if $\forall x \in B(xRb)$. This means that $b$ is related to every other element in $B$ by the relation $R$, with $b$ on the right side, indicating it is "larger" than every other element in $B$. Note that if there is a largest element, it is unique. Note that it’d also be the only maximal element. Note that if there is a largest element, it is the lowest upper bound.},
}

\newglossaryentry{maximal}{
    name={maximal},
    description={Suppose $R$ is a partial order on a set $A$, $B \subseteq A$, and $b \in B$. Then $b$ is called an $R$-maximal element (or just a maximal element if $R$ is clear from the context) if $\lnot \exists x \in B(bRx \land x \neq b)$. This means that there does not exist an element $x \in B$ such that $b$ is related to $x$ by the relation $R$ with $b$ on the left side (indicating $b$ is "smaller"), and $x$ is not equal to $b$.},
}

\newglossaryentry{lower bound}{
    name={lower bound},
    description={Suppose $R$ is a partial order on $A$, $B \subseteq A$, and $a \in A$. Then $a$ is called a lower bound for $B$ if $\forall x \in B(aRx)$. This means that $a$ is related to every element in $B$ by the relation $R$, with $a$ on the left side, indicating it is "smaller" than every element in $B$. Note that a lower bound for $B$ need not be an element of $B$, which is the only difference between lower bounds and smallest elements.},
}

\newglossaryentry{upper bound}{
    name={upper bound},
    description={Suppose $R$ is a partial order on $A$, $B \subseteq A$, and $a \in A$. Then $a$ is called an upper bound for $B$ if $\forall x \in B(xRa)$. This means that $a$ is related to every element in $B$ by the relation $R$, with $a$ on the right side, indicating it is "larger" than every element in $B$. Note that an upper bound for $B$ need not be an element of $B$, which is the only difference between upper bounds and largest elements.},
}

\newglossaryentry{least upper bound}{
    name={least upper bound},
    description={Suppose $R$ is a partial order on $A$ and $B \subseteq A$. Let $U$ be the set of all upper bounds for $B$. If $U$ has a smallest element, then this smallest element is called the least upper bound of $B$. The least upper bound is sometimes abbreviated as l.u.b.},
}

\newglossaryentry{greatest lower bound}{
    name={greatest lower bound},
    description={Suppose $R$ is a partial order on $A$ and $B \subseteq A$. Let $L$ be the set of all lower bounds for $B$. If $L$ has a largest element, then this largest element is called the greatest lower bound of $B$. The greatest lower bound is sometimes abbreviated as g.l.b.},
}

\newglossaryentry{equivalence relation}{
    name={equivalence relation},
    description={Suppose $R$ is a relation on a set $A$. Then $R$ is called an equivalence relation on $A$ (or just an equivalence relation if $A$ is clear from context) if it is reflexive, symmetric, and transitive.},
}

\newglossaryentry{equivalence class}{
    name={equivalence class},
    description={Given a set $A$ with a relation $R$ on $A$, an equivalence class for $R$ is a subset of $A$ consisting of all elements that are related to each other by $R$. More formally, for an element $a \in A$, the equivalence class of $a$ under $R$ is the set $\{x \in A \mid aRx\}$. Note that different equivalence classes are disjoint. In other words, equivalence classes are either identical or disjoint (i.e. given set $A$ where $x,y \in A$, if $[x]$ and $[y]$ are identical, then we're referring to the same equivalence class. If they're not the same equivalence class, then they're disjoint). Also note that all equivalence classes are nonempty.},
    plural={equivalence classes},
}

\newglossaryentry{partition}{
    name={partition},
    description={A partition of a set $A$ is a family $\mathcal{F}$ of non-empty subsets of $A$ such that:
        \begin{enumerate}
            \item The union of all subsets in $\mathcal{F}$ equals $A$: $\bigcup \mathcal{F} = A$.
            \item The subsets in $\mathcal{F}$ are pairwise disjoint: $\forall X, Y \in \mathcal{F}$ with $X \neq Y$, $X \cap Y = \varnothing$.
        \end{enumerate}
        This means that every element of $A$ belongs to exactly one subset in $\mathcal{F}$.},
}

\newglossaryentry{pairwise disjoint}{
    name={pairwise disjoint},
    description={A family $\mathcal{F}$ of sets is pairwise disjoint if every pair of distinct sets in $\mathcal{F}$ are disjoint. Formally, $\mathcal{F}$ is pairwise disjoint if $\forall X, Y \in \mathcal{F}$, with $X \neq Y$, it holds that $X \cap Y = \varnothing$.},
}

\newglossaryentry{quotient set}{
    name={quotient set},
    description={Suppose $A$ is a set with an equivalence relation $R$. The set of all equivalence classes of elements of $A$ under $R$ is called the \textit{quotient set} and is denoted by $A/R$. Formally,
        \[
        A/R = \{[x]_{R} \mid x \in A\} = \{X \subseteq A \mid \exists x \in A(X = [x]_{R})\}.
        \]
        This means that $A/R$ is the set containing all equivalence classes $[x]_{R}$ of elements $x$ in $A$.},
}

\newglossaryentry{lemma}{
    name={lemma},
    description={A fact that is proven primarily for the purpose of being used to prove a theorem. Lemmas are intermediate propositions or auxiliary theorems that assist in establishing a larger result.}
}

\newglossaryentry{congruent}{
    name={congruent},
    description={Given a positive integer $m$, two integers $x$ and $y$ are said to be congruent modulo $m$ if there exists an integer $k$ such that $x - y = km$. In other words, $x$ is congruent to $y$ modulo $m$ if and only if $m$ divides $(x - y)$. This is denoted as $x \equiv y \, (\text{mod } m)$. Two numbers being congruent modulo $m$ means they have the same remainder when divided by $m$.}
}

\newglossaryentry{function}{
    name={function},
    description={A function $F$ from a set $A$ to a set $B$ is a relation such that for every element $a \in A$, there exists exactly one element $b \in B$ such that the ordered pair $(a, b) \in F$. Formally, $F$ is a function from $A$ to $B$ if:
        \[
        \forall a \in A \exists! b \in B ((a, b) \in F).
        \]
    This ensures that each input from the domain $A$ is associated with a unique output in the codomain $B$. Functions are often denoted as $F \text{: } A \to B$.}
}

\newglossaryentry{image}{
    name={image},
    description={Suppose $f$ is a function from a set $A$ to a set $B$. The image of an element $a \in A$ under $f$ is the second coordinate of the ordered pair $(a, f(a))$ in the relation that defines the function. In other words, if $f: A \to B$, then $f(a) \in B$ is the image of $a \in A$. The range of $f$, which is the set of all possible images of elements of $A$ under $f$, is given by:
        \[
        \text{Ran}(f) = \{f(a) \mid a \in A\}.
        \]
        This set contains all elements in $B$ that are mapped to by some element in $A$.}
}

\newglossaryentry{one-to-one}{
    name={one-to-one},
    description={
        Suppose $f \text{: } A \to B$. The function $f$ is said to be one-to-one if for any two elements $a_{1}, a_{2} \in A$, whenever $f(a_{1}) = f(a_{2})$ it must follow that $a_{1} = a_{2}$. In other words, no two distinct elements of $A$ map to the same element in $B$. Formally:
        \[
        \lnot \exists a_{1} \in A \exists a_{2} \in A(f(a_{1}) = f(a_{2}) \land a_{1} \neq a_{2}).
        \]
    }
}

\newglossaryentry{onto}{
    name={onto},
    description={
        Suppose $f \text{: } A \to B$. The function $f$ maps onto $B$ (or is onto) if every element of $B$ is the image of at least one element in $A$. That is, for every $b \in B$, there exists an $a \in A$ such that $f(a) = b$. Formally:
        \[
        \forall b \in B \exists a \in A(f(a) = b).
        \]
    }
}

\newglossaryentry{injection}{
    name={injection},
    description={An injection is another term for a \gls{one-to-one} function.}
}

\newglossaryentry{surjection}{
    name={surjection},
    description={A surjection is another term for an \gls{onto} function.}
}

\newglossaryentry{one-to-one correspondence}{
    name={one-to-one correspondence},
    description={A function $f \text{: } A \to B$ is called a one-to-one correspondence if it is both one-to-one and onto. This means that $f$ maps each element of $A$ to a unique element of $B$ and covers all elements of $B$. In other words, $f$ is both an injection and a surjection.}
}

\newglossaryentry{bijection}{
    name={bijection},
    description={A bijection is another term for a one-to-one correspondence.}
}

\newglossaryentry{sufficient}{
    name={sufficient},
    description={A condition $P$ is called sufficient for a condition $Q$ if $P$ implies $Q$. In other words, whenever $P$ is true, $Q$ must also be true.}
}

\newglossaryentry{necessary}{
    name={necessary},
    description={A condition $Q$ is called necessary for a condition $P$ if $P$ implies $Q$. In other words, for $P$ to be true, $Q$ must also be true.}
}

\newglossaryentry{converse}{
    name={converse},
    description={The converse of a statement of the form "if $P$, then $Q$" is the statement "if $Q$, then $P$". The converse of a statement does not always have the same truth value as the original statement.}
}

% ==== SYMB GLOSSARY ====
\newglossaryentry{symb:therefore}{
	name=$\therefore$,
	description={therefore, which is used for an argument's conclusion.},
	type=symbolslist
}
\newglossaryentry{symb:not}{
	name=$\lnot$,
	description={a logical negation. Note that this symbol always applies only to the statement that comes immediately after it.},
	type=symbolslist
}
\newglossaryentry{symb:and}{
	name=$\land$,
	description={a logical AND.},
	type=symbolslist
}
\newglossaryentry{symb:inclusive or}{
	name=$\lor$,
	description={a logical OR. Note that this is the inclusive or.},
	type=symbolslist
}
\newglossaryentry{symb:exclusive or}{
	name=$\oplus$,
	description={a logical OR. Note that this is the exclusive or.},
	type=symbolslist
}
\newglossaryentry{symb:in}{
    name=$\in$,
    description={is an element of.},
    type=symbolslist
}

\newglossaryentry{symb:not in}{
    name=$\notin$,
    description={is not an element of.},
    type=symbolslist
}

\newglossaryentry{symb:where}{
    name=$\mid$,
    description={abbreviation used in set construction to define the statements that must be true for a given variable to belong to the set in question.},
    type=symbolslist
}

\newglossaryentry{symb:real number set}{
    name=$\mathbb{R}$,
    description={The set of all \glspl{real number}.},
    type=symbolslist
}

\newglossaryentry{symb:rational number set}{
    name=$\mathbb{Q}$,
    description={The set of all \glspl{rational number}.},
    type=symbolslist
}

\newglossaryentry{symb:irrational number set}{
    name=$\mathbb{I}$,
    description={The set of all \glspl{irrational number}.},
    type=symbolslist
}

\newglossaryentry{symb:integer number set}{
    name=$\mathbb{Z}$,
    description={The set of all \glspl{integer number}.},
    type=symbolslist
}

\newglossaryentry{symb:natural number set}{
    name=$\mathbb{N}$,
    description={The set of all \glspl{natural number}. Note that some mathematicians consider $0$ as a natural number and some don't.},
    type=symbolslist
}

\newglossaryentry{symb:complex number set}{
    name=$\mathbb{C}$,
    description={The set of all \glspl{complex number}.},
    type=symbolslist
}

\newglossaryentry{symb:empty set}{
    name=$\varnothing$,
    description={The empty set, which contains no elements.},
    type=symbolslist
}

\newglossaryentry{symb:intersection}{
    name=$\cap$,
    description={Set intersection. $A \cap B = \{x \text{ \gls{symb:where} } x \in A \text{ and } x \in B\}$.},
    type=symbolslist
}

\newglossaryentry{symb:union}{
    name=$\cup$,
    description={Set union. $A \cup B = \{x \text{ \gls{symb:where} } x \in A \text{ or } x \in B\}$.},
    type=symbolslist
}

\newglossaryentry{symb:difference}{
    name=$\setminus$,
    description={Set difference. $A \setminus B = \{x \text{ \gls{symb:where} } x \in A \text{ and } x \notin B\}$.},
    type=symbolslist
}

\newglossaryentry{symb:symmetric difference}{
    name=$\triangle$,
    description={Set symmetric difference. $A \triangle B = (A \setminus B) \cup (B \setminus A) = \{x \mid (x \in A \text{ and } x \notin B) \text{ or } (x \in B \text{ and } x \notin A)\}$.},
    type=symbolslist
}

\newglossaryentry{symb:subset of}{
    name=$\subseteq$,
    description={A is a subset of B, $A \subseteq B$, if every element of A is also an element of B. Logical form: $\forall x(x \in A \rightarrow x \in B)$},
    type=symbolslist
}

\newglossaryentry{symb:implies}{
    name=$\rightarrow$,
    description={$P \rightarrow Q$ means "if $P$, then $Q$". It is read as "P implies Q". $P$ is the \gls{antecedent} and $Q$ is the \gls{consequent}.},
    type=symbolslist
}

\newglossaryentry{symb:for all}{
    name=$\forall$,
    description={For all, universal quantifier. $\forall x \in A$, read as "for all $x$ in $A$". Used to refer to \textit{every} value of $x$. Note that this quantifier binds $x$. Note that this symbol always applies only to the statement that comes immediately after it.},
    type=symbolslist
}

\newglossaryentry{symb:there exists}{
    name=$\exists$,
    description={There exists, existential quantifier. $\exists x \in A$, read as "there exists an $x$ in $A$". Used to refer to \textit{at least one} value of $x$. Note that this quantifier binds $x$. Note that this symbol always applies only to the statement that comes immediately after it.},
    type=symbolslist
}

\newglossaryentry{symb:there exists!}{
    name=$\exists!$,
    description={There exists exactly one. $\exists! x \in A$, read as "there exists exactly one $x$ in $A$". Used to refer to \textit{one and only one} value of $x$. Note that this quantifier binds $x$. Note that this symbol always applies only to the statement that comes immediately after it.},
    type=symbolslist
}

\newglossaryentry{symb:iff}{
    name=$\leftrightarrow$,
    description={$P \leftrightarrow Q$ means "P if and only if Q". It indicates that $P$ is true exactly when $Q$ is true, and vice versa.},
    type=symbolslist
}

\newglossaryentry{symb:power set}{
    name=$\mathcal{P}$,
    description={The power set},
    type=symbolslist
}

\newglossaryentry{symb:intersection of a family of sets}{
    name=$\bigcap$,
    description={Intersection of a family of sets. $\bigcap_{i \in I} A_i$ represents the intersection of the sets $A_i$ for all $i \in I$, i.e., $\bigcap_{i \in I} A_i = \{x \mid \forall i \in I, x \in A_i\}$.},
    type=symbolslist
}

\newglossaryentry{symb:union of a family of sets}{
    name=$\bigcup$,
    description={Union of a family of sets. $\bigcup_{i \in I} A_i$ represents the union of the sets $A_i$ for all $i \in I$, i.e., $\bigcup_{i \in I} A_i = \{x \mid \exists i \in I, x \in A_i\}$.},
    type=symbolslist
}

\newglossaryentry{symb:divides}{
    name=$\mid$,
    description={divides. $x \mid y$ means $x$ divides $y$, i.e., $\exists k \in \mathbb{Z}(kx = y)$.},
    type=symbolslist
}

\newglossaryentry{symb:composition}{
    name=$\circ$,
    description={Composition of relations. If $R$ is a relation from $A$ to $B$ and $S$ is a relation from $B$ to $C$, then the \gls{composition} $S \circ R$ is the relation from $A$ to $C$ defined as $S \circ R = \{(a,c) \in A \times C \mid \exists b \in B((a,b) \in R \text{ and } (b,c) \in S)\}$.},
    type=symbolslist
}

\newglossaryentry{symb:congruent}{
    name={$\equiv$},
    description={Denotes congruence in modular arithmetic. Specifically, $ x \equiv y \, (\text{mod } m) $ means that integers $x$ and $y$ are congruent modulo $m$, which is equivalent to stating that $m$ divides the difference $x - y$. This symbol is used to indicate that two numbers have the same remainder when divided by $m$.},
    type=symbolslist
}

\newglossaryentry{symb:function}{
    type=symbolslist,
    name={$\to$},
    description={Denotes a function from one set to another. Specifically, $F \text{: } A \to B$ indicates that $F$ is a function from the set $A$ to the set $B$, meaning that every element in $A$ is mapped to exactly one element in $B$.}
}
