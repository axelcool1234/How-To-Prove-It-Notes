\newglossary[slg]{symbolslist}{syi}{syg}{Symbols}
% ==== MAIN GLOSSARY ====
\newglossaryentry{conjecture}{
	name={conjecture},
	description={what mathematicians call their guesses, similarly to how scientists call their guesses hypotheses.}
}

\newglossaryentry{theorem}{
	name={theorem},
	description={conjectures that have been proven. A theorem is a statement that has been proven to be true based on a set of assumptions (hypotheses). It asserts that if the hypotheses are true, then a specific conclusion must also be true.}
}

\newglossaryentry{hypothesis}{
    name={hypothesis},
    description={In the context of a theorem, a hypothesis is an assumption or a set of assumptions that are stated within the theorem. If the hypotheses are true, they support the truth of the theorem's conclusion.},
    plural={hypotheses}
}

\newglossaryentry{instance}{
    name={instance},
    description={An instance of a theorem is a specific case obtained by assigning particular values to the free variables within the theorem. For all instances where the hypotheses are true, the conclusion must also be true for the theorem to be correct.}
}

\newglossaryentry{counterexample}{
    name={counterexample},
    description={A counterexample to a theorem is a specific instance where the hypotheses of the theorem are true, but the conclusion is false. The existence of a counterexample disproves the theorem.}
}

\newglossaryentry{deductive reasoning}{
	name={deductive reasoning},
	description={the foundation on which proofs are based. Proofs play a central role in mathematics.}
}

\newglossaryentry{argument}{
	name={argument},
	description={a set of premises and a conclusion. An argument is valid if it is the case that, if the premises are true then the conclusion must be true.}
}

\newglossaryentry{premise}{
	name={premise},
	description={a fact or assumption in an argument.}
}

\newglossaryentry{conclusion}{
	name={conclusion},
	description={the end product of an argument. In a valid argument the conclusion will be true if the given premises are true.}
}

\newglossaryentry{connective symbol}{
	name={connective symbol},
	description={stand-ins for some of the words used to combine statements in an argument.}	
}

\newglossaryentry{conjunction}{
	name={conjunction},
	description={P \gls{symb:and} Q is the conjunction of P and Q.}
}

\newglossaryentry{disjunction}{
	name={disjunction},
	description={P \gls{symb:inclusive or} Q is the disjunction of P and Q.}
}

\newglossaryentry{negation}{
	name={negation},
	description={\gls{symb:not} P is the negation of P.}
}

\newglossaryentry{conditional}{
	name={conditional},
    description={P \gls{symb:implies} Q is a conditional.}
}

\newglossaryentry{well-formed formula}{
	name={well-formed formula},
	description={a "grammatical" expression in the language of logic.}
}

\newglossaryentry{free variable}{
    name={free variable},
    description={The free variables in a statement stand for objects that the statement says something about. Plugging in different values for a free variable affects the meaning of a statement and may change its truth value. The fact that you can plug in different values for a free variable means that it is free to stand for anything.}
}

\newglossaryentry{bound variable}{
    name={bound variable},
    description={The bound variables (also known as dummy variables) in a statement are simply letters that are used as a convenience to help express an idea and should not be thought of as standing for any particular object. A bound variable can always be replaced by a new variable without changing the meaning of the statement, and often the statement can be rephrased so that the bound variables are eliminated altogether.}
}

\newglossaryentry{truth set}{
    name={truth set},
    description={The set of all values that make a given satement true.}
}

\newglossaryentry{universe of discourse}{
    name={universe of discourse},
    description={The set of all possible values that \glspl{free variable} can take in a given statement. The universe of discourse is often denoted by $U$.}
}

\newglossaryentry{rational number}{
    name={rational number},
    description={A number that can be expressed as the quotient of two integers, where the denominator is not zero. Rational numbers are denoted by $\mathbb{Q}$.}
}

\newglossaryentry{irrational number}{
    name={irrational number},
    description={A number that cannot be expressed as the quotient of two integers. Irrational numbers have non-repeating, non-terminating decimal expansions and include numbers such as $\pi$ and $\sqrt{2}$.}
}

\newglossaryentry{integer number}{
    name={integer number},
    description={A number that can be written without a fractional or decimal component. Integers include the set of positive whole numbers, negative whole numbers, and zero, and are denoted by $\mathbb{Z}$.}
}

\newglossaryentry{real number}{
    name={real number},
    description={A value that represents a quantity along a continuous line. Real numbers include all rational numbers (such as fractions and integers) and all irrational numbers (numbers that cannot be expressed as a fraction of two integers), and are denoted by $\mathbb{R}$.}
}

\newglossaryentry{natural number}{
    name={natural number},
    description={A positive integer (1, 2, 3, ...) used for counting and ordering. Some definitions include zero, while others do not. Natural numbers are denoted by $\mathbb{N}$.}
}

\newglossaryentry{complex number}{
    name={complex number},
    description={A number in the form $a + bi$, where $a$ and $b$ are real numbers, and $i$ is the imaginary unit with the property that $i^2 = -1$. The real part of the complex number is $a$, and the imaginary part is $b$.}
}

\newglossaryentry{disjoint}{
    name={disjoint},
    description={Sets $A$ and $B$ are said to be disjoint if $A \cap B = \varnothing$, which means that $A$ and $B$ have no common elements.},
}

\newglossaryentry{antecedent}{
    name={antecedent},
    description={In a conditional statement $P \rightarrow Q$, the antecedent is $P$. It is the statement that comes before the arrow and is assumed to be true for the implication to hold.}
}

\newglossaryentry{consequent}{
    name={consequent},
    description={In a conditional statement $P \rightarrow Q$, the consequent is $Q$. It is the statement that comes after the arrow and is asserted to be true if the \gls{antecedent} $P$ is true.}
}

\newglossaryentry{arbitrary}{
    name={arbitrary},
    description={In the context of a proof, a variable is considered arbitrary if no specific assumptions are made about its value. An arbitrary variable can represent any element within the \gls{universe of discourse}, and the proof must hold for all possible values of this variable.}
}

% Logical Equivalences
\newglossaryentry{De Morgan's laws}{
    name={De Morgan's laws},
    description={
        $\neg(P \land Q)$ is equivalent to $\neg P \lor \neg Q$.\\
        $\neg(P \lor Q)$ is equivalent to $\neg P \land \neg Q$.
    }
}
\newglossaryentry{Commutative laws}{
    name={Commutative laws},
    description={
        $P \land Q$ is equivalent to $Q \land P$.\\
        $P \lor Q$ is equivalent to $Q \lor P$.
    }
}
\newglossaryentry{Associative laws}{
    name={Associative laws},
    description={
        $P \land (Q \land R)$ is equivalent to $(P \land Q) \land R$.\\
        $P \lor (Q \lor R)$ is equivalent to $(P \lor Q) \lor R$.
    }
}
\newglossaryentry{Idempotent laws}{
    name={Idempotent laws},
    description={
        $P \land P$ is equivalent to $P$.\\
        $P \lor P$ is equivalent to $P$.
    }
}
\newglossaryentry{Distributive laws}{
    name={Distributive laws},
    description={
        $P \land (Q \lor R)$ is equivalent to $(P \land Q) \lor (P \land R)$.\\
        $P \lor (Q \land R)$ is equivalent to $(P \lor Q) \land (P \lor R)$.
    }
}
\newglossaryentry{Absorption laws}{
    name={Absorption laws},
    description={
        $P \lor (P \land Q)$ is equivalent to $P$.\\
        $P \land (P \lor Q)$ is equivalent to $P$.
    }
}
\newglossaryentry{Double Negation laws}{
    name={Double Negation laws},
    description={
        $\neg \neg P$ is equivalent to $P$.
    }
}
\newglossaryentry{Tautology laws}{
    name={Tautology laws},
    description={Formulas that are always true are called tautologies.\\
        $P \land \text{(a tautology)}$ is equivalent to $P$.\\
        $P \lor \text{(a tautology)}$ is a tautology.\\
        $\neg \text{(a tautology)}$ is a contradiction. 
    }
}
\newglossaryentry{Contradiction laws}{
    name={Contradiction laws},
    description={Formulas that are always false are called contradictions.\\
        $P \land \text{(a contradiction)}$ is a contradiction.\\
        $P \lor \text{(a contradiction)}$ is equivalent to $P$.\\
        $\neg \text{(a contradiction)}$ is a tautology.
    }
}

\newglossaryentry{Conditional laws}{
    name={Conditional laws},
    description={
        $P \text{ \gls{symb:implies} } Q$ is equivalent to $\neg P \lor Q$.\\
        $P \text{ \gls{symb:implies} } Q$ is equivalent to $\neg (P \land \neg Q)$.
    }
}

\newglossaryentry{Contrapositive law}{
    name={Contrapositive law},
    description={
        $P \text{ \gls{symb:implies} } Q$ is equivalent to $\neg Q \rightarrow \neg P$. 
    }
}

\newglossaryentry{Quantifier Negation laws}{
    name={Quantifier Negation laws},
    description={
        $\neg \exists x P(x)$ is equivalent to $\forall x \neg P(x)$.\\
        $\neg \forall x P(x)$ is equivalent to $\exists x \neg P(x)$.\\
    }
}

\newglossaryentry{vacuously true}{
    name={vacuously true},
    description={A statement is said to be vacuously true if it asserts that all elements of the empty set have a certain property. For example, the statement "all unicorns have wings" is vacuously true because there are no unicorns. In general, a statement of the form $\forall x \in A, P(x)$ is vacuously true if $A$ is empty. Another way of looking at it is converting the universal statement that is vacuously true into a conditional statement of the form $\forall x (x \in A \rightarrow P(x))$ and realizing that the antecedent cannot be satisfied (which in our example would be the statement $x \in A$. This is because there is no $x$ at all in our set of unicorns - it is the empty set).}
}

\newglossaryentry{family of sets}{
    name={family of sets},
    description={A family of sets is a set whose elements are themselves sets. It is often indexed by another set, allowing each set in the family to be identified by a unique index. For example, $\{A_i \mid i \in I\}$ denotes a family of sets $\{A_i\}$ indexed by $I$.}
}

\newglossaryentry{power set}{
    name={power set},
    description={The power set of a set $A$, denoted by $\text{\gls{symb:power set}}(A)$ or $2^A$, is the set of all subsets of $A$, including the empty set and $A$ itself. Formally, $\mathcal{P}(A) = \{B \mid B \subseteq A\}$.}
}

\newglossaryentry{given}{
    name={given},
    description={In the context of a proof, givens are the statements that are known or assumed to be true at a particular point in the proof process. Initially, the givens include the hypotheses of the theorem being proven, but they may later include other statements inferred from the hypotheses or added as new assumptions through transformations.}
}

\newglossaryentry{goal}{
    name={goal},
    description={In the context of a proof, the goal is the statement that remains to be proven at a particular point in the proof process. Initially, the goal is the conclusion of the theorem being proven, but it may change several times as transformations are applied during the course of the proof.}
}

\newglossaryentry{existential instantiation}{
    name={existential instantiation},
    description={A logical rule that allows one to introduce a new variable to stand for an object that satisfies an existential statement. If you have a given of the form $\exists x P(x)$, you can introduce a new variable $x_0$ to represent an object for which $P(x_0)$ is true, thereby allowing the assumption that $P(x_0)$ is true.}
}

\newglossaryentry{universal instantiation}{
    name={universal instantiation},
    description={A logical rule that allows one to derive a statement about a specific element from a universally quantified statement. If you have a given of the form $\forall x P(x)$, you can substitute any specific value, say $a$, for $x$ and conclude that $P(a)$ is true, allowing you to treat $P(a)$ as a new given.}
}

\newglossaryentry{modus ponens}{
    name={modus ponens},
    description={A logical rule that states if $P \rightarrow Q$ is true and $P$ is true, then $Q$ must also be true. Also known as the rule of detachment.}
}

\newglossaryentry{modus tollens}{
    name={modus tollens},
    description={A logical rule that states if $P \rightarrow Q$ is true and $Q$ is false, then $P$ must also be false. Also known as the rule of contrapositive.}
}

\newglossaryentry{case}{
    name={case},
    description={When one of your givens is a disjunction, such as $P \lor Q$, you need to consider two possibilities: if $P$ is true and if $Q$ is true. This involves creating separate arguments for each possibility to show that both lead to the goal.},
}

\newglossaryentry{exhaustive}{
    name={exhaustive},
    description={When all possible cases are considered to ensure that every scenario is accounted for. Mathematicians consider an argument \textit{exhaustive} if it includes all potential situations, ensuring completeness of the proof.}
}

\newglossaryentry{disjunctive syllogism}{
    name={disjunctive syllogism},
    description={A rule of inference stating that if you have a given of $P \lor Q$ and also a given of $\lnot P$, then you can conclude that $Q$ is true (adding it to your givens). Similarly, if you have a given of $P \lor Q$ and also a given of $\lnot Q$, then you can conclude that $P$ is true (adding it to your givens).}
}

\newglossaryentry{coordinate}{
    name={coordinate},
    description={An element in an ordered pair or tuple. In an ordered pair $(a, b)$, $a$ is the first coordinate and $b$ is the second coordinate. The order of the coordinates makes a difference in the pair or tuple.}
}

\newglossaryentry{singleton}{
    name={singleton},
    description={A set that contains exactly one element. For example, $\{a\}$ is a singleton set containing the element $a$.}
}

% ==== SYMB GLOSSARY ====
\newglossaryentry{symb:therefore}{
	name=$\therefore$,
	description={therefore, which is used for an argument's conclusion.},
	type=symbolslist
}
\newglossaryentry{symb:not}{
	name=$\lnot$,
	description={a logical negation. Note that this symbol always applies only to the statement that comes immediately after it.},
	type=symbolslist
}
\newglossaryentry{symb:and}{
	name=$\land$,
	description={a logical AND.},
	type=symbolslist
}
\newglossaryentry{symb:inclusive or}{
	name=$\lor$,
	description={a logical OR. Note that this is the inclusive or.},
	type=symbolslist
}
\newglossaryentry{symb:exclusive or}{
	name=$\oplus$,
	description={a logical OR. Note that this is the exclusive or.},
	type=symbolslist
}
\newglossaryentry{symb:in}{
    name=$\in$,
    description={is an element of.},
    type=symbolslist
}

\newglossaryentry{symb:not in}{
    name=$\notin$,
    description={is not an element of.},
    type=symbolslist
}

\newglossaryentry{symb:where}{
    name=$\mid$,
    description={abbreviation used in set construction to define the statements that must be true for a given variable to belong to the set in question.},
    type=symbolslist
}

\newglossaryentry{symb:real number set}{
    name=$\mathbb{R}$,
    description={The set of all \glspl{real number}.},
    type=symbolslist
}

\newglossaryentry{symb:rational number set}{
    name=$\mathbb{Q}$,
    description={The set of all \glspl{rational number}.},
    type=symbolslist
}

\newglossaryentry{symb:irrational number set}{
    name=$\mathbb{I}$,
    description={The set of all \glspl{irrational number}.},
    type=symbolslist
}

\newglossaryentry{symb:integer number set}{
    name=$\mathbb{Z}$,
    description={The set of all \glspl{integer number}.},
    type=symbolslist
}

\newglossaryentry{symb:natural number set}{
    name=$\mathbb{N}$,
    description={The set of all \glspl{natural number}. Note that some mathematicians consider $0$ as a natural number and some don't.},
    type=symbolslist
}

\newglossaryentry{symb:complex number set}{
    name=$\mathbb{C}$,
    description={The set of all \glspl{complex number}.},
    type=symbolslist
}

\newglossaryentry{symb:empty set}{
    name=$\varnothing$,
    description={The empty set, which contains no elements.},
    type=symbolslist
}

\newglossaryentry{symb:intersection}{
    name=$\cap$,
    description={Set intersection. $A \cap B = \{x \text{ \gls{symb:where} } x \in A \text{ and } x \in B\}$.},
    type=symbolslist
}

\newglossaryentry{symb:union}{
    name=$\cup$,
    description={Set union. $A \cup B = \{x \text{ \gls{symb:where} } x \in A \text{ or } x \in B\}$.},
    type=symbolslist
}

\newglossaryentry{symb:difference}{
    name=$\setminus$,
    description={Set difference. $A \setminus B = \{x \text{ \gls{symb:where} } x \in A \text{ and } x \notin B\}$.},
    type=symbolslist
}

\newglossaryentry{symb:symmetric difference}{
    name=$\triangle$,
    description={Set symmetric difference. $A \triangle B = (A \setminus B) \cup (B \setminus A) = \{x \mid (x \in A \text{ and } x \notin B) \text{ or } (x \in B \text{ and } x \notin A)\}$.},
    type=symbolslist
}

\newglossaryentry{symb:subset of}{
    name=$\subseteq$,
    description={A is a subset of B, $A \subseteq B$, if every element of A is also an element of B. Logical form: $\forall x(x \in A \rightarrow x \in B)$},
    type=symbolslist
}

\newglossaryentry{symb:implies}{
    name=$\rightarrow$,
    description={$P \rightarrow Q$ means "if $P$, then $Q$". It is read as "P implies Q". $P$ is the \gls{antecedent} and $Q$ is the \gls{consequent}.},
    type=symbolslist
}

\newglossaryentry{symb:for all}{
    name=$\forall$,
    description={For all, universal quantifier. $\forall x \in A$, read as "for all $x$ in $A$". Used to refer to \textit{every} value of $x$. Note that this quantifier binds $x$. Note that this symbol always applies only to the statement that comes immediately after it.},
    type=symbolslist
}

\newglossaryentry{symb:there exists}{
    name=$\exists$,
    description={There exists, existential quantifier. $\exists x \in A$, read as "there exists an $x$ in $A$". Used to refer to \textit{at least one} value of $x$. Note that this quantifier binds $x$. Note that this symbol always applies only to the statement that comes immediately after it.},
    type=symbolslist
}

\newglossaryentry{symb:there exists!}{
    name=$\exists!$,
    description={There exists exactly one. $\exists! x \in A$, read as "there exists exactly one $x$ in $A$". Used to refer to \textit{one and only one} value of $x$. Note that this quantifier binds $x$. Note that this symbol always applies only to the statement that comes immediately after it.},
    type=symbolslist
}

\newglossaryentry{symb:iff}{
    name=$\leftrightarrow$,
    description={$P \leftrightarrow Q$ means "P if and only if Q". It indicates that $P$ is true exactly when $Q$ is true, and vice versa.},
    type=symbolslist
}

\newglossaryentry{symb:power set}{
    name=$\mathcal{P}$,
    description={The power set},
    type=symbolslist
}

\newglossaryentry{symb:intersection of a family of sets}{
    name=$\bigcap$,
    description={Intersection of a family of sets. $\bigcap_{i \in I} A_i$ represents the intersection of the sets $A_i$ for all $i \in I$, i.e., $\bigcap_{i \in I} A_i = \{x \mid \forall i \in I, x \in A_i\}$.},
    type=symbolslist
}

\newglossaryentry{symb:union of a family of sets}{
    name=$\bigcup$,
    description={Union of a family of sets. $\bigcup_{i \in I} A_i$ represents the union of the sets $A_i$ for all $i \in I$, i.e., $\bigcup_{i \in I} A_i = \{x \mid \exists i \in I, x \in A_i\}$.},
    type=symbolslist
}

\newglossaryentry{symb:divides}{
    name=$\mid$,
    description={divides. $x \mid y$ means $x$ divides $y$, i.e., $\exists k \in \mathbb{Z}(kx = y)$.},
    type=symbolslist
}
