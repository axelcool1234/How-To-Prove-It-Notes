\documentclass{article}
\usepackage{amssymb}
\usepackage{amsmath}
\usepackage{gensymb}
\usepackage{enumerate}
\usepackage{tikz}
\usetikzlibrary{positioning}
\usepackage[
nonumberlist, %do not show page numbers
nopostdot,    %do not add additional periods
acronym,      %generate acronym listing
toc,          %show listings as entries in table of contents
section]      %use section level for toc entries
{glossaries}
\usepackage[automake]{glossaries-extra}

\loadglsentries{glossary}
\makeglossaries

\newcommand{\printGls}[1]{%
    \textbf{\Gls{#1}} - \glsentrydesc{#1}%
}
\newcommand{\printGlspl}[1]{%
    \textbf{\Glspl{#1}} - \glsentrydesc{#1}%
}
\newcommand{\textiff}{\text{ iff }}

% PROOF SCRATCHWORK COMMANDS (Used to create Tikz givens/goals diagrams).
% Counter for automatic naming
\newcounter{proofboxnum}

\newcommand{\beginProofBoxes}[2]{
    % Initialize counter and first two boxes
    \setcounter{proofboxnum}{1}
    \node[draw, rectangle, text width=5cm, align=center] (given\theproofboxnum) {\textbf{Givens:}\\#1};
    \node[draw, rectangle, text width=5cm, align=center, right=0.5cm of given\theproofboxnum] (goal\theproofboxnum) {\textbf{Goals:}\\#2};
        
    % Increment the box number
    \stepcounter{proofboxnum}
}

\newcommand{\addProofBoxes}[2]{
    % Calculate previous box numbers
    \pgfmathtruncatemacro{\previousgiven}{\theproofboxnum - 1}
    \pgfmathtruncatemacro{\previousgoal}{\theproofboxnum - 1}
    
    % Add new boxes
    \node[draw, rectangle, text width=5cm, align=center, below=0.5cm of given\previousgiven] (given\theproofboxnum) {\textbf{Givens:}\\#1};
    \node[draw, rectangle, text width=5cm, align=center, right=0.5cm of given\theproofboxnum] (goal\theproofboxnum) {\textbf{Goals:}\\#2};
    
    % Draw arrows
    \draw[->] (given\previousgiven) -- (given\theproofboxnum);
    \draw[->] (goal\previousgoal) -- (goal\theproofboxnum);
     
    % Increment the box number
    \stepcounter{proofboxnum}
}


\title{Exercises}
\author{Axel Sorenson}
\date{August 5th, 2024}

\begin{document}
\maketitle
\section{Setential Logic}
\section{Quantificational Logic}
\section{Proofs}
\subsection{Proof Strategies}
\begin{enumerate}
\setcounter{enumi}{7} 
\item Suppose $A \setminus B \subseteq C \cap D$ and $x \in A$. Prove that if $x \notin D$ then $x \in B$.\\

    \noindent Suppose $A \setminus B \subseteq C \cap D$ and $x \in A$. Suppose $x \notin D$. That means $x \notin C \cap D$. Since $A \setminus B \subseteq C \cap D$, this means $x \notin A \setminus B$. This is equivalent to $\forall x \in A \rightarrow x \in B$. Since $x \in A$, then $x \in B$. Thus, if $x \notin D$, then $x \in B$.

\end{enumerate}

\subsection{Proof Involving Negations and Conditionals}
\begin{enumerate}
\item 
    \begin{enumerate}
    \item Suppose $P \rightarrow Q$ and $Q \rightarrow R$ are both true. Prove that $P \rightarrow R$ is true.\\

        \noindent Suppose $P \rightarrow Q$ and $Q \rightarrow R$. Suppose $P$. Given that $P$ is true, then $Q$ must be true. Given that $Q$ is true, then $R$ must be true. Thus, if $P$ is true, then $R$ is true.

    \item Suppose $\lnot R \rightarrow (P \rightarrow \lnot Q)$ is true. Prove that $P \rightarrow (Q \rightarrow R)$ is true.\\

        \noindent Suppose $\lnot R \rightarrow (P \rightarrow \lnot Q)$. Suppose $\lnot R$. That means $P \rightarrow \lnot Q$ is true. Suppose $P$ is true. That means $\lnot Q$ is true. Given that $\lnot R$ leads to $\lnot Q$ being true $(\lnot R \rightarrow \lnot Q)$ if $P$ is true, then the contrapositive is true: $Q \rightarrow R$. Since $P$ is assumed to be true, this shows that $P \rightarrow (Q \rightarrow R)$ is true.
    \end{enumerate}
\setcounter{enumi}{4} 
\item Prove that it cannot be the case that $x \in A \setminus B$ and $x \in B \setminus C$.\\

\noindent For the sake of contradiction, suppose $x \in A \setminus B$ and $x \in B \setminus C$. This means, for $x \in A \setminus B$ to be true, that $x \in A$ and $x \notin B$. For $x \in B \setminus C$ to be true, $x \in B$ and $x \notin C$. It has just been shown that $x \in B$ and $x \notin B$, which is a contradiction. Thus, it cannot be the case that $x \in A \setminus B$ and $x \in B \setminus C$.
\end{enumerate}

\subsection{Proof Involving Quantifiers} 
\begin{enumerate}
\setcounter{enumi}{8}
\item Prove that if $\mathcal{F}$ is a family of sets and $A \in F$, then $\bigcap \mathcal{F} \subseteq A$.\\

\noindent Suppose $A \in \mathcal{F}$. Lets split this into cases. Suppose $\bigcap \mathcal{F} = \varnothing$. Since the empty set is a subet of every set, this means $\bigcap \mathcal{F} \subseteq A$. Now suppose $\bigcap \mathcal{F} \neq \varnothing$. This means $\exists x \in \bigcap \mathcal{F}$. Based on the definition of $\bigcap$, $\forall x \forall B((x \in \bigcap \mathcal{F} \land B \in \mathcal{F}) \rightarrow x \in B)$. Now, since $A \in \mathcal{F}$, all elements in $\bigcap \mathcal{F}$ are also in $A$. Is it possible for there to be an element in $\bigcap \mathcal{F}$ but not in $A$? For the sake of contradiction, lets assume that to be the case. But by the definition of $\bigcap$, every element in $\bigcap \mathcal{F}$ is also an element of every set in $\mathcal{F}$. So this'd be a contradiction. Therefore, $\bigcap \mathcal{F} \subseteq A$.\\

\noindent Cleaned up version:\\
Suppose $\mathcal{F}$ is a family of sets and $A \in F$. Consider two cases. If $\bigcap \mathcal{F} = \varnothing$, then by definition $\varnothing \subseteq A$. Since the empty set is a subset of every set, $\bigcap \mathcal{F} \subseteq A$ holds. If $\bigcap \mathcal{F} \neq \varnothing$, then there exists an element $x \in \bigcap \mathcal{F}$. By the definition of intersection, $\forall x \forall B((x \in \bigcap \mathcal{F} \land B \in \mathcal{F}) \rightarrow x \in B)$. Since $A \in \mathcal{F}$, it follows that $x \in A$ for all $x \in \bigcap \mathcal{F}$. Hence, every element of $\bigcap \mathcal{F}$ is also an element of $A$, so $\bigcap \mathcal{F} \subseteq A$. In both cases, we conclude that $\bigcap \mathcal{F} \subseteq A$.

\end{enumerate}

\subsection{Proof Involving Conjunctions and Biconditionals} 
\subsection{Proof Involving Disjunctions} 
\subsection{Existence and Uniqueness Proofs}
\subsection{More Examples of Proofs}


\clearpage
\printglossary[type=\acronymtype,style=long]  % list of acronyms
\printglossary[type=symbolslist,style=long]   % list of symbols
\printglossary[type=main]                     % main glossary
\end{document}
