\documentclass{article}
\usepackage{amssymb}
\usepackage{amsmath}
\usepackage{gensymb}
\usepackage{enumerate}
\usepackage[
nonumberlist, %do not show page numbers
nopostdot,    %do not add additional periods
acronym,      %generate acronym listing
toc,          %show listings as entries in table of contents
section]      %use section level for toc entries
{glossaries}
\usepackage[automake]{glossaries-extra}

\loadglsentries{glossary}
\makeglossaries

\newcommand{\printGls}[1]{%
    \textbf{\Gls{#1}} - \glsentrydesc{#1}%
}
\newcommand{\printGlspl}[1]{%
    \textbf{\Glspl{#1}} - \glsentrydesc{#1}%
}


\title{Exercises}
\author{Axel Sorenson}
\date{August 5th, 2024}

\begin{document}
\maketitle
\section{Setential Logic}
\section{Quantificational Logic}
\section{Proofs}
\subsection{Proof Strategies}
\subsection{Proof Involving Negations and Conditionals}
\begin{enumerate}
\item 
    \begin{enumerate}
    \item Suppose $P \rightarrow Q$ and $Q \rightarrow R$ are both true. Prove that $P \rightarrow R$ is true.\\

        \noindent Suppose $P \rightarrow Q$ and $Q \rightarrow R$. Suppose $P$. Given that $P$ is true, then $Q$ must be true. Given that $Q$ is true, then $R$ must be true. Thus, if $P$ is true, then $R$ is true.

    \item Suppose $\lnot R \rightarrow (P \rightarrow \lnot Q)$ is true. Prove that $P \rightarrow (Q \rightarrow R)$ is true.\\

        \noindent Suppose $\lnot R \rightarrow (P \rightarrow \lnot Q)$. Suppose $\lnot R$. That means $P \rightarrow \lnot Q$ is true. Suppose $P$ is true. That means $\lnot Q$ is true. Given that $\lnot R$ leads to $\lnot Q$ being true $(\lnot R \rightarrow \lnot Q)$ if $P$ is true, then the contrapositive is true: $Q \rightarrow R$. Since $P$ is assumed to be true, this shows that $P \rightarrow (Q \rightarrow R)$ is true.
    \end{enumerate}
\end{enumerate}
\subsection{Proof Involving Quantifiers} 
\subsection{Proof Involving Conjunctions and Biconditionals} 
\subsection{Proof Involving Disjunctions} 
\subsection{Existence and Uniqueness Proofs}
\subsection{More Examples of Proofs}


\clearpage
\printglossary[type=\acronymtype,style=long]  % list of acronyms
\printglossary[type=symbolslist,style=long]   % list of symbols
\printglossary[type=main]                     % main glossary
\end{document}
