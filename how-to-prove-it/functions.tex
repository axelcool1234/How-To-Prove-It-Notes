\documentclass{article}
\usepackage{amssymb}
\usepackage{amsmath}
\usepackage{gensymb}
\usepackage{enumerate}
\usepackage[
nonumberlist, %do not show page numbers
nopostdot,    %do not add additional periods
acronym,      %generate acronym listing
toc,          %show listings as entries in table of contents
section]      %use section level for toc entries
{glossaries}
\usepackage[automake]{glossaries-extra}

\loadglsentries{glossary}
\makeglossaries

\newcommand{\printGls}[1]{%
    \textbf{\Gls{#1}} - \glsentrydesc{#1}%
}
\newcommand{\printGlspl}[1]{%
    \textbf{\Glspl{#1}} - \glsentrydesc{#1}%
}


\title{Functions}
\author{Axel Sorenson}
\date{August 4th, 2024}

\begin{document}
\maketitle

\section{Functions}
Suppose $F$ is a relation from $A$ to $B$. Then $F$ is called a \gls{function} from $A$ to $B$ if for every $a \in A$ there is exactly one $b \in B$ such that $(a, b) \in F$. In other words, to say that $F$ is a function from $A$ to $B$ means:
\begin{center}
    $\forall a \in A \exists!b \in B((a, b) \in F)$.
\end{center}
To indicate that $F$ is a function from $A$ to $B$, we will write $F \text{: } A \text{ \gls{symb:function} } B$.\\

\noindent Suppose $f \text{: } A \to B$. $\forall a \in A \text{ and } \forall b \in B$, $b = f(a) \textiff (a, b) \in f$.\\

\noindent If $A$ is any set and $a \in A$, then $(a, a) \in i_{A}$, so $i_{A}(a) = a$.\\

\noindent A function $f$ from a set $A$ to another set $B$ is often specified by giving a rule that can be used to determine $f(a)$ for any $a \in A$.\\

\noindent Any subset of $A \times B$ that satisfies the formal definition of a function above is a function from $A$ to $B$.\\

\noindent Suppose $f$ and $g$ are functions from $A$ to $B$. If $\forall a \in A(f(a) = g(a))$, then $f = g$. This means, to prove that two functions $f$ and $g$ are equal, we must prove $\forall a \in A(f(a) = g(a))$.\\

\noindent Suppose $f$ is a function from $A$ to $B$. The second coordinate of an ordered pair in $f$ is what we have called the \gls{image} of its first coordinate. Thus, the range of $f$ could also be described as the set of all images of elements of $A$ under $f$:
\begin{center}
    $\text{Ran}(f) = \{f(a) \mid a \in A \}$.
\end{center}
As for the domain of $f$: the subset of $A$ for the domain of $f$ happens to be all of $A$ since every element of $A$ must appear as the coordinate of one ordered pair in $f$.\\

\noindent The definition of \gls{composition} of relations can also be applied to functions. If $f \text{: } A \to B$ and $g \text{: } B \to C$, then $f$ is a relation from $A$ to $B$ and $g$ is a relation from $B$ to $C$, so $g \circ f$ will be a relation from $A$ to $C$. In fact, it turns out that $g \circ f$ is a function from $A$ to $C$, as the next theorem shows:
\begin{center}
Suppose $f \text{: } A \to B$ and $g \text{: } B \to C$. Then $g \circ f \text{: } A \to C$, and for any $a \in A$, the value of $g \circ f$ at $a$ is given by the formula $(g \circ f)(a) = g(f(a))$.
\end{center}
$(g \circ f)(a) = c = g(b) = g(f(a))$.

\section{One-to-One and Onto}
The following are two properties of functions:\\

\noindent Suppose $f \text{: } A \to B$. We will say that $f$ is \gls{one-to-one} if:
\begin{center}
    $\lnot \exists a_{1} \in A \exists a_{2} \in A(f(a_{1}) = f(a_{2}) \land a_{1} \neq a_{2})$.
\end{center}
We say that $f$ maps \gls{onto} $B$ (or just is onto if $B$ is clear from context) if:
\begin{center}
    $\forall b \in B \exists a \in A(f(a) = b)$.
\end{center}
One-to-one functions are sometimes called \glspl{injection}, and onto functions are sometimes called \glspl{surjection}.\\

\noindent Suppose $f \text{: } A \to B$
\begin{enumerate}
    \item $f$ is one-to-one$\textiff \forall a_{1} \in A \forall a_{2} \in A(f(a_{1}) = f(a_{2}) \rightarrow a_{1} = a_{2})$.
    \item $f$ is onto$\textiff \text{Ran}(f) = B$.
\end{enumerate}

\noindent Suppose $f \text{: } A \to B$ and $g \text{: } B \to C$. It follows that $g \circ f \text{: } A \to C$.
\begin{enumerate}
    \item If $f$ and $g$ are both one-to-one, then so is $g \circ f$.
    \item If $f$ and $g$ are both onto, then so is $g \circ f$.
\end{enumerate}
Functions that are both one-to-one and onto are particularly important in mathematics. Such functions are sometimes called \glspl{one-to-one correspondence} or \glspl{bijection}. If there is a one-to-one correspondence between two finite sets, then the sets must have the same number of elements. This is one of the reasons why one-to-one correspondences are so important. 

\section{Inverses of Functions}
The inverse of a function from $A$ to $B$ is not always a function from $B$ to $A$. For example, an element of $B$ may not be paired with an element of $A$. Additionally, an element of $B$ may be paired more than once with elements of $A$. Both of these scenarios leads to a failure to meet the requirements of the definition of being a function. You may have noticed that the reasons why $f^{-1}$ isn't a function from $B$ to $A$ are related to the reasons why $f$ is neither one-to-one nor onto.\\

\noindent Suppose $f \text{: } A \to B$. If $f$ is one-to-one and onto, then $f^{-1} \text{: } B \to A$. If $f^{-1}$ is a function from $B$ to $A$, then $f$ must be one-to-one and onto. \\

\noindent The composition of a function and its inverse (given that the inverse is also a function), results in the identity function. To better see this, suppose $f \text{: } A \to B$ and $f^{-1} \text{: } B \to A$.
\begin{center}
    $(f^{-1} \circ f)(a) = f^{-1}(f(a)) = f^{-1}(b) = a = i_{A}(a)$.
$(f \circ f^{-1})(b) = f(f^{-1}(b)) = f(a) = b = i_{B}(b)$.
\end{center}
Suppose $f \text{: } A \to B$
\begin{enumerate}
    \item If there is a function $g \text{: } B \to A$ such that $g \circ f = i_{A}$ then $f$ is one-to-one.
    \item If there is a function $g \text{: } B \to A$ such that $f \circ g = i_{B}$ then $f$ is onto.
\end{enumerate}
Suppose $f \text{: } A \to B$. Then the following statements are equivalent:
\begin{enumerate}
    \item $f$ is one-to-one and onto.
    \item $f^{-1} \text{: } B \to A$.
    \item There is a function $g \text{: } B \to A$ such that $g \circ f = i_{A}$ and $f \circ g = i_{B}$.
\end{enumerate}
Interesting: The function $g(x) = \log x$ (assuming base 10) is the inverse of the function $f(x) = 10^{x}$.

\section{Closures}
Often in mathematics we work with a function from a set to itself. In that situation, the following concept can be useful:
\begin{center}
Suppose $f \text{: } A \to A$ and $C \subseteq A$. We will say that $C$ is \gls{closed} under $f$ if $\forall x \in C(f(x) \in C)$.
\end{center}
Suppose $f \text{: } A \to A$ and $B \subseteq A$. Then the \gls{closure} of $B$ under $f$ is the smallest set $C \subseteq A$ such that $B \subseteq C$ and $C$ is closed under $f$, if there is such a smallest set. In other words, a set $C \subseteq A$ is the closure of $B$ under $f$ if it has the following properties:
\begin{enumerate}
    \item $B \subseteq C$.
    \item $C$ is closed under $f$.
    \item For every set $D \subseteq A$, if $B \subseteq D$ and $D$ is closed under $f$ then $C \subseteq D$.
\end{enumerate}
If a set has a smallest element, then it can have only one smallest element. Similarly, if a set $A$ has a closure under a function $f$, then this closure must be unique, so it makes sense to call it \textit{the} closure rather than \textit{a} closure.\\

\noindent Suppose $f \text{: } A \times A \to A$ and $C \subseteq A$. We will say that $C$ is closed under $f$ if $\forall x \in C \forall y \in C(f(x,y) \in C)$.\\

\noindent A function from $A \times A$ to $A$ could be thought of as an operation that can be applied to a pair of objects $(x, y) \in A \times A$ to produce another element of $A$. Often in mathematics an operation to be performed on a pair of mathematical objects $(x, y)$ is represented by a symbol that we write between $x$ and $y$. For example, if $x$ and $y$ are real numbers then $x + y$ denotes another number, and if $x$ and $y$ are sets then $x \cup y$ is another set. Imitating this notation, when mathematicians define a function from $A \times A$ to $A$ they sometimes represent it with a symbol rather than a letter, and they write the result of applying the function to a pair $(x, y)$ by putting the symbol between $x$ and $y$, rather than by putting a letter before $(x, y)$. When a function from $A \times A$ to $A$ is written in this way, it is usually called a \gls{binary operation} on $A$.

\clearpage
\printglossary[type=\acronymtype,style=long]  % list of acronyms
\printglossary[type=symbolslist,style=long]   % list of symbols
\printglossary[type=main]                     % main glossary
\end{document}
