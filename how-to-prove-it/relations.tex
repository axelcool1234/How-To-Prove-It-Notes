\documentclass{article}
\usepackage{amssymb}
\usepackage{amsmath}
\usepackage{gensymb}
\usepackage{enumerate}
\usepackage[
nonumberlist, %do not show page numbers
nopostdot,    %do not add additional periods
acronym,      %generate acronym listing
toc,          %show listings as entries in table of contents
section]      %use section level for toc entries
{glossaries}
\usepackage[automake]{glossaries-extra}

\loadglsentries{glossary}
\makeglossaries

\newcommand{\printGls}[1]{%
    \textbf{\Gls{#1}} - \glsentrydesc{#1}%
}
\newcommand{\printGlspl}[1]{%
    \textbf{\Glspl{#1}} - \glsentrydesc{#1}%
}


\title{Relations}
\author{Axel Sorenson}
\date{July 31st, 2024}

\begin{document}
\maketitle

\section{Ordered Pairs and Cartesian Products}
When making a truth set for a statement with more than one free variables, the truth set will not contain individual values, but rather ordered pairs of values. The order of values in an ordered pair makes a difference. That means $(6, 18)$ is not the same as $(18, 6)$. When referring to $a$, $b$, $c$, etc. in an ordered pair/triple/quadruple/etc. $(a, b, c, ...)$, we call them \glspl{coordinate}. $a$ is the first coordinate, $b$ is the second, $c$ is the third, etc.\\

\noindent Suppose $A$ and $B$ are sets. Then the Cartesian product of $A$ and $B$, denoted $A \times B$, is the set of all ordered pairs in which the first coordinate is an element of $A$ and the second is an element of $B$. In other words,
\begin{center}
$A \times B = \{(a,b) \mid a \in A \text{ and } b \in B\}$.
\end{center}

\section{Relations}
Suppose $P(x, y)$ is a statement with two free variables $x$ and $y$. Often such a statement can be thought of as expressing a \gls{relationship} between $x$ and $y$. The truth set of the statement $P(x, y)$ is a set of ordered pairs that records when this relationship holds. In fact, it is often useful to think of any set of ordered pairs in this way, as a record of when some relationship holds. This is the motivation behind the following definition:
\begin{center}
    Suppose $A$ and $B$ are sets. Then a set $R \subseteq A \times B$ is called a \gls{relation} from $A$ to $B$.
\end{center}
The following are definitions regarding relationships:\\
Suppose $R$ is a relation from $A$ to $B$. Then the \gls{domain} of $R$ is the set
\begin{center}
    $\text{Dom}(R) = \{a \in A \mid \exists b \in B((a,b) \in R)\}$.
\end{center}
The \gls{range} of $R$ is the set
\begin{center}
    $\text{Ran}(R) = \{b \in B \mid \exists a \in A((a,b) \in R)\}$.
\end{center}
The \gls{inverse} of $R$ is the set
\begin{center}
    $R^{-1} = \{(b,a) \in B \times A \mid (a,b) \in R\}$.
\end{center}
Finally, suppose $R$ is a relation from $A$ to $B$ and $S$ is a relation from $B$ to $C$. Then the \gls{composition} of $S$ and $R$ is the relation $S \text{ \gls{symb:composition} } R$ from $A$ to $C$ defined as
\begin{center}
    $S \circ R = \{(a,c) \in A \times C \mid \exists b \in B((a,b) \in R \text{ and } (b,c) \in S)\}$.
\end{center}

\bigskip
\noindent The domain of a relation from $A$ to $B$ is the set containing all the first coordinates of ordered pairs in the relation. This in general will be a subset of $A$, but it need not be all of $A$.\\

\noindent The range of a relation from $A$ to $B$ is the set containing all of the second coordinates of ordered pairs in the relation. This in general will be a subset of $B$, but it need not be all of $B$.\\

\noindent The inverse of a relation contains exactly the same ordered pairs as the original relation, but with the order of the coordinates of each pair reversed.\\

\noindent The inverse of the inverse of a relation is equivalent to the relation itself: $(R^{-1})^{-1} = R$.\\ 

\noindent The domain of the inverse of a relation is equivalent to the range of the relation: $\text{Dom}(R^{-1}) = \text{Ran}(R)$. This works in the reverse, where the range of the inverse of a relation is equivalent to the domain of the relation: $\text{Ran}(R^{-1}) = \text{Dom}(R)$.\\

\noindent Compositions are associative, but not commutative. So $T \circ (S \circ R)= (T \circ S) \circ R$ and $A \circ B \neq B \circ A$.

\section{More About Relations}
The notation for a relationship between object $x$ and object $y$ given $R$ is a relation from $A$ to $B$ and $x \in A$ and $y \in B$, $xRy$ means $(x,y) \in R$. This mirrors mathematical relationships such as $x = y$, $x < y$, $x \in y$, $x \subseteq y$, etc. which involves a mathematical symbol between $x$ and $y$ to indicate a relationship between the objects.\\

\noindent A relation $R$ from set $A$ to the same set $A$ is called a \gls{relation on} $A$. For these relations (a relation from a set to the same set), instead of drawing two circles (representing sets), vertices (representing objects within the sets), and edges (representing objects belonging to ordered pairs in the relation), Mathematicians often represent these relations with \glspl{directed graph}.\\

\noindent Some definitions:\\
Suppose $R$ is a relation on $A$.
\begin{enumerate}
    \item $R$ is said to be \gls{reflexive} on $A$ (or just reflexive, if $A$ is clear from context) if $\forall x \in A(xRx)$, or in other words $\forall x \in A((x,x) \in R)$.
    \item $R$ is said to be \gls{symmetric} if $\forall x \in A \forall y \in A(xRy \rightarrow yRx)$.
    \item $R$ is said to be \gls{transitive} if $\forall x \in A \forall y \in A \forall z \in A((xRy \land yRz) \rightarrow xRz)$.
    \item $R$ is said to be \gls{antisymmetric} if $\forall x \in A \forall y \in A((xRy \land yRx) \rightarrow x = y)$
\end{enumerate}
If a relation is reflexive, then its directed graph will have \glspl{loop} on all its vertices.\\

\noindent Suppose $A$ is a set. The \gls{identity relation} $i_{A} = \{(x,y) \in A \times A \mid x = y \}$. The identity relation could also be defined as $i_{A} = \{(x,x) \mid x \in A \}$. The identity relation $i_{A}$ for any set $A$ will be reflexive, symmetric, and transitive.\\

\noindent The following connects the properties and operations defined about sets so far:\\
Suppose $R$ is a relation on $A$.
\begin{enumerate}
    \item $R$ is reflexive$\textiff i_{A} \subseteq R$, where as before $i_{A}$ is the identity relation on $A$.
    \item $R$ is symmetric$\textiff R = R^{-1}$.
    \item $R$ is transitive$\textiff R \circ R \subseteq R$.
\end{enumerate}

\section{Ordering Relations}
Suppose $R$ is a relation on a set $A$. Then $R$ is called a \gls{partial order} on $A$ (or just a partial order if $A$ is clear from context) if it is reflexive, transitive, and antisymmetric. It is called a \gls{total order} on $A$ (or just a total order if $A$ is clear from context) if it is a partial order, and in addition it has the following propery:
\begin{center}
    $\forall x \in A \forall y \in A(xRy \lor yRx)$.
\end{center}
Some definitions:\\
Suppose $R$ is a partial order on a set $A$, $B \subseteq A$, and $b \in B$. Then $b$ is called an R-\gls{smallest} of $B$ (or just a smallest element if $R$ is clear from the context) if $\forall x \in B(bRx)$. It is called an R-\gls{minimal} element (or just a minimal element if $R$ is clear from the context) if $\lnot \exists x \in B(xRb \land x \neq b)$. Similary, $b$ is called an R-\gls{largest} of $B$ (or just a largest element if $R$ is clear from the context) if $\forall x \in B(xRb)$. It is called an R-\gls{maximal} element (or just a maximal element if $R$ is clear from the context) if $\lnot \exists x \in B(bRx \land x \neq b)$.\\

\noindent Suppose $R$ is a partial order on $A$, $B \subseteq A$, and $a \in A$. Then $a$ is called a \gls{lower bound} for $B$ if $\forall x \in B(aRx)$. Similarly, it is an \gls{upper bound} for $B$ if $\forall x \in B(xRa)$.\\

\noindent Suppose $R$ is a partial order on $A$ and $B \subseteq A$. Let $U$ be the set of all upper bounds for $B$, and let $L$ be the set of all lower bounds. If $U$ has a smallest element, then this smallest element is called the \gls{least upper bound} of $B$. If $L$ has a largest element, then this largest element is called the \gls{greatest lower bound} of $B$. The phrases least upper bound and greatest lower bound are sometimes abbreviated l.u.b. and g.l.b.

\section{Equivalence Relations}
Suppose $R$ is a relation on a set $A$. Then $R$ is called an \gls{equivalence relation} on $A$ (or just an equivalence relation if $A$ is clear from context) if it is reflexive, symmetric, and transitive.\\

\noindent Given a set $A$ with a relation $R$ on $A$, there are \glspl{equivalence class} for that relation such that every element in $A$ is an element of exactly one of the equivalence classes. The relation $R$ consists of those pairs $(p, q) \in A \times A$ such that the elements $p$ and $q$ are in the same equivalence class. We can form a family $\mathcal{F}$ of subsets of $A$ where each subset is one of these equivalence classes. Given that this breaks up $A$ into disjoint pieces, we can call the family $\mathcal{F}$ a \gls{partition} of $A$. It turns out that every equivalence relation on a set $A$ determines a partition of $A$, whose elements are the equivalence classes for the equivalence relation.\\

\noindent A family $\mathcal{F}$ is \gls{pairwise disjoint} if every pair of distince elements of $\mathcal{F}$ are disjoint. In other words, $\forall X \in \mathcal{F} \forall Y \in \mathcal{F}(X \neq Y \rightarrow X \cap Y = \varnothing)$.\\

\noindent To be more formal, a family $\mathcal{F}$ is a partition of $A$ if
\begin{enumerate}
    \item $\bigcup \mathcal{F} = A$.
    \item $\mathcal{F}$ is pairwise disjoint.
    \item $\forall X \in \mathcal{F}(X \neq \varnothing)$.
\end{enumerate}

\noindent To be more formal: Suppose $R$ is an equivalence relation on a set $A$, and $x \in A$. Then the equivalance class of $x$ with respect to $R$ is the set
\begin{center}
    $[x]_{R} = \{y \in A \mid yRx\}$.
\end{center}
If $R$ is clear from context, then we write $[x]$ instead of $[x]_{R}$. The set of all equivalence classes of elements of $A$ is called $A$ modulo $R$, and is called $A$ \textit{modulo} $R$, and is denoted $A/R$. Thus,
\begin{center}
$A/R = \{[x]_{R} \mid x \in A\} = \{X \subseteq A \mid \exists x \in A(X = [x]_{R})\}$.
\end{center}
This is known as the \gls{quotient set}.\\

\noindent Facts that are proven primarily for the purpose of using them to prove a theorem are usually called \glspl{lemma}.\\

\noindent Suppose $m$ is a positive integer. For any integers $x$ and $y$, we will say that $x$ is \gls{congruent} to $y$ modulo $m$ if $\exists k \in Z(x - y = km)$. In other words, $x$ is congruent to $y$ modulo $m \textiff m \mid (x - y)$. We will use the notation $x \text{ \gls{symb:congruent} } y (\text{mod } m)$ to mean that $x$ is congruent to $y$ modulo $m$.

\clearpage
\printglossary[type=\acronymtype,style=long]  % list of acronyms
\printglossary[type=symbolslist,style=long]   % list of symbols
\printglossary[type=main]                     % main glossary
\end{document}
