\documentclass{article}
\usepackage{amssymb}
\usepackage{amsmath}
\usepackage{gensymb}
\usepackage{enumerate}
\usepackage[
nonumberlist, %do not show page numbers
nopostdot,    %do not add additional periods
acronym,      %generate acronym listing
toc,          %show listings as entries in table of contents
section]      %use section level for toc entries
{glossaries}
\usepackage[automake]{glossaries-extra}

\loadglsentries{glossary}
\makeglossaries

\newcommand{\printGls}[1]{%
    \textbf{\Gls{#1}} - \glsentrydesc{#1}%
}
\newcommand{\printGlspl}[1]{%
    \textbf{\Glspl{#1}} - \glsentrydesc{#1}%
}


\title{Relations}
\author{Axel Sorenson}
\date{July 31st, 2024}

\begin{document}
\maketitle

\section{Ordered Pairs and Cartesian Products}
When making a truth set for a statement with more than one free variables, the truth set will not contain individual values, but rather ordered pairs of values. The order of values in an ordered pair makes a difference. That means $(6, 18)$ is not the same as $(18, 6)$. When referring to $a$, $b$, $c$, etc. in an ordered pair $(a, b, c, ...)$, we call them \glspl{coordinate}. $a$ is the first coordinate, $b$ is the second, etc.\\

\noindent Suppose $A$ and $B$ are sets. Then the Cartesian product of $A$ and $B$, denoted $A \times B$, is the set of all ordered pairs in which the first coordinate is an element of $A$ and the second is an element of $B$. In other words,
\begin{center}
$A \times B = \{(a,b) \mid a \in A \text{ and } b \in B\}$.
\end{center}


\clearpage
\printglossary[type=\acronymtype,style=long]  % list of acronyms
\printglossary[type=symbolslist,style=long]   % list of symbols
\printglossary[type=main]                     % main glossary
\end{document}
